%    EdoTeX is a TeX setup for Education
%    Copyright (C) 2010-2016  Benjamin 'Benno' Falkner
%   
%    Permission is hereby granted, free of charge, to any person obtaining
%    a copy of this software and associated documentation files 
%    (the "Software"), to deal in the Software without restriction, including
%    without limitation the rights to use, copy, modify, merge, publish, 
%    distribute, sublicense, and/or sell copies of the Software, and to permit
%    persons to whom the Software is furnished to do so, subject to the 
%    following conditions:
%
%    The above copyright notice and this permission notice shall be included 
%    in all copies or substantial portions of the Software.
%
%    THE SOFTWARE IS PROVIDED "AS IS", WITHOUT WARRANTY OF ANY KIND, EXPRESS 
%    OR IMPLIED, INCLUDING BUT NOT LIMITED TO THE WARRANTIES OF 
%    MERCHANTABILITY, FITNESS FOR A PARTICULAR PURPOSE AND NONINFRINGEMENT. 
%    IN NO EVENT SHALL THE AUTHORS OR COPYRIGHT HOLDERS BE LIABLE FOR ANY 
%    CLAIM, DAMAGES OR OTHER LIABILITY, WHETHER IN AN ACTION OF CONTRACT, 
%    TORT OR OTHERWISE, ARISING FROM, OUT OF OR IN CONNECTION WITH THE 
%    SOFTWARE OR THE USE OR OTHER DEALINGS IN THE SOFTWARE.


% plain TeX default file -- is replaced 

\catcode`\{=1 % left brace is begin-group character
\catcode`\}=2 % right brace is end-group character
\catcode`\$=3 % dollar sign is math shift
\catcode`\&=4 % ampersand is alignment tab
\catcode`\#=6 % hash mark is macro parameter character
\catcode`\^=7 \catcode`\^^K=7 % circumflex and uparrow are for superscripts
\catcode`\_=8 \catcode`\^^A=8 % underline and downarrow are for subscripts
\catcode`\^^I=10 % ascii tab is a blank space
\chardef\active=13 \catcode`\~=\active % tilde is active
\catcode`\^^L=\active \outer\def^^L{\par} % ascii form-feed is "\outer\par"


% Here is a list of the characters that have been specially catcoded:
\def\dospecials{\do\ \do\\\do\{\do\}\do\$\do\&%
  \do\#\do\^\do\^^K\do\_\do\^^A\do\%\do\~}

\catcode`@=11

\pdfoutput=1

\newlinechar=`@ 
\message{
    EdoTeX  Copyright (C) 2010 - 2016  Benjamin 'Benno' Falkner@
    This program comes with ABSOLUTELY NO WARRANTY@
    This is free software, and you are welcome to redistribute it@
    under certain conditions.@}


\message{EdoTex loading core element: codes,}
%Based on plain TeX

% INITEX sets up \mathcode x=x, for x=0..255, except that
% \mathcode x=x+"7100, for x = `A to `Z and `a to `z;
% \mathcode x=x+"7000, for x = `0 to `9.
% The following changes define internal codes as recommended
% in Appendix C of The TeXbook:
\mathcode`\^^@="2201 % \cdot
\mathcode`\^^A="3223 % \downarrow
\mathcode`\^^B="010B % \alpha
\mathcode`\^^C="010C % \beta
\mathcode`\^^D="225E % \land
\mathcode`\^^E="023A % \lnot
\mathcode`\^^F="3232 % \in
\mathcode`\^^G="0119 % \pi
\mathcode`\^^H="0115 % \lambda
\mathcode`\^^I="010D % \gamma
\mathcode`\^^J="010E % \delta
\mathcode`\^^K="3222 % \uparrow
\mathcode`\^^L="2206 % \pm
\mathcode`\^^M="2208 % \oplus
\mathcode`\^^N="0231 % \infty
\mathcode`\^^O="0140 % \partial
\mathcode`\^^P="321A % \subset
\mathcode`\^^Q="321B % \supset
\mathcode`\^^R="225C % \cap
\mathcode`\^^S="225B % \cup
\mathcode`\^^T="0238 % \forall
\mathcode`\^^U="0239 % \exists
\mathcode`\^^V="220A % \otimes
\mathcode`\^^W="3224 % \leftrightarrow
\mathcode`\^^X="3220 % \leftarrow
\mathcode`\^^Y="3221 % \rightarrow
\mathcode`\^^Z="8000 % \ne
\mathcode`\^^[="2205 % \diamond
\mathcode`\^^\="3214 % \le
\mathcode`\^^]="3215 % \ge
\mathcode`\^^^="3211 % \equiv
\mathcode`\^^_="225F % \lor
\mathcode`\ ="8000 % \space
\mathcode`\!="5021
\mathcode`\'="8000 % ^\prime
\mathcode`\(="4028
\mathcode`\)="5029
\mathcode`\*="2203 % \ast
\mathcode`\+="202B
\mathcode`\,="613B
\mathcode`\-="2200
\mathcode`\.="013A
\mathcode`\/="013D
\mathcode`\:="303A
\mathcode`\;="603B
\mathcode`\<="313C
\mathcode`\=="303D
\mathcode`\>="313E
\mathcode`\?="503F
\mathcode`\[="405B
\mathcode`\\="026E % \backslash
\mathcode`\]="505D
\mathcode`\_="8000 % \_
\mathcode`\{="4266
\mathcode`\|="026A
\mathcode`\}="5267
\mathcode`\^^?="1273 % \smallint

% for uppercase letters. The following changes are needed:
\sfcode`\)=0 \sfcode`\'=0 \sfcode`\]=0

% Finally, INITEX sets all \delcode values to -1, except \delcode`.=0
\delcode`\(="028300
\delcode`\)="029301
\delcode`\[="05B302
\delcode`\]="05D303
\delcode`\<="26830A
\delcode`\>="26930B
\delcode`\/="02F30E
\delcode`\|="26A30C
\delcode`\\="26E30F
% N.B. { and } should NOT get delcodes; otherwise parameter grouping fails!


% To make the plain macros more efficient in time and space,
% several constant values are declared here as control sequences.
% If they were changed, anything could happen; so they are private symbols.
\chardef\@ne=1
\chardef\tw@=2
\chardef\thr@@=3
\chardef\sixt@@n=16
\chardef\@cclv=255
\mathchardef\@cclvi=256
\mathchardef\@m=1000
\mathchardef\@M=10000
\mathchardef\@MM=20000

% Additional UTF8 support
%will be enhanced to support more european languages 
\catcode`^^c2=13 \catcode`^^c3=13 \catcode`^^e2=13
\def^^c2#1#2{\expandafter\def\csname c2:#1\endcsname{#2}}
\def^^c3#1#2{\expandafter\def\csname c3:#1\endcsname{#2}}
\def^^e2#1#2#3{\expandafter\def\csname e2:#1#2\endcsname{#3}}
ß{\char223}
ä{\char228}
ö{\char246}
ü{\char252}
Ä{\char196}
Ö{\char214}
Ü{\char220}
Æ{\char198}
æ{\char230}
å{\char229}
Å{\r{A}}
%ø 
%Ø
%Œ
%œ
€{Euro} %  Eurosign is not part of the  T1 font encoding and will be added later (now replacement text is set to avoid errors)
\def^^c2#1{\csname c2:#1\endcsname}
\def^^c3#1{\csname c3:#1\endcsname}
\def^^e2#1#2{\csname e2:#1#2\endcsname}

\message{registers,}

\count10=22 % allocates \count registers 23, 24, ...
\count11=9 % allocates \dimen registers 10, 11, ...
\count12=9 % allocates \skip registers 10, 11, ...
\count13=9 % allocates \muskip registers 10, 11, ...
\count14=9 % allocates \box registers 10, 11, ...
\count15=9 % allocates \toks registers 10, 11, ...
\count16=-1 % allocates input streams 0, 1, ...
\count17=-1 % allocates output streams 0, 1, ...
\count18=3 % allocates math families 4, 5, ...
\count19=0 % allocates \language codes 1, 2, ...
\count20=255 % allocates insertions 254, 253, ...
\countdef\insc@unt=20 % the insertion counter
\countdef\allocationnumber=21 % the most recent allocation
\countdef\m@ne=22 \m@ne=-1 % a handy constant
\def\wlog{\immediate\write\m@ne} % write on log file (only)

% Here are abbreviations for the names of scratch registers
% that don't need to be allocated.

\countdef\count@=255
\dimendef\dimen@=0
\dimendef\dimen@i=1 % global only
\dimendef\dimen@ii=2
\skipdef\skip@=0
\toksdef\toks@=0

%Allocation
\outer\def\newcount{\alloc@0\count\countdef\insc@unt}
\outer\def\newdimen{\alloc@1\dimen\dimendef\insc@unt}
\outer\def\newskip{\alloc@2\skip\skipdef\insc@unt}
\outer\def\newmuskip{\alloc@3\muskip\muskipdef\@cclvi}
\outer\def\newbox{\alloc@4\box\chardef\insc@unt}
\let\newtoks=\relax % we do this to allow plain.tex to be read in twice
\outer\def\newhelp#1#2{\newtoks#1#1\expandafter{\csname#2\endcsname}}
\outer\def\newtoks{\alloc@5\toks\toksdef\@cclvi}
\outer\def\newread{\alloc@6\read\chardef\sixt@@n}
\outer\def\newwrite{\alloc@7\write\chardef\sixt@@n}
\outer\def\newfam{\alloc@8\fam\chardef\sixt@@n}
\outer\def\newlanguage{\alloc@9\language\chardef\@cclvi}
\def\alloc@#1#2#3#4#5{\global\advance\count1#1by\@ne
  \ch@ck#1#4#2% make sure there's still room
  \allocationnumber=\count1#1%
  \global#3#5=\allocationnumber
  \wlog{\string#5=\string#2\the\allocationnumber}}
\outer\def\newinsert#1{\global\advance\insc@unt by\m@ne
  \ch@ck0\insc@unt\count
  \ch@ck1\insc@unt\dimen
  \ch@ck2\insc@unt\skip
  \ch@ck4\insc@unt\box
  \allocationnumber=\insc@unt
  \global\chardef#1=\allocationnumber
  \wlog{\string#1=\string\insert\the\allocationnumber}}
\def\ch@ck#1#2#3{\ifnum\count1#1<#2%
  \else\errmessage{No room for a new #3}\fi}


% Here are some examples of allocation.
\newdimen\maxdimen \maxdimen=16383.99999pt % the largest legal <dimen>
\newskip\hideskip \hideskip=-1000pt plus 1fill % negative but can grow
\newskip\centering \centering=0pt plus 1000pt minus 1000pt
\newdimen\p@ \p@=1pt % this saves macro space and time
\newdimen\z@ \z@=0pt % can be used both for 0pt and 0
\newskip\z@skip \z@skip=0pt plus0pt minus0pt
\newbox\voidb@x % permanently void box register

% And here's a different sort of allocation:
% For example, \newif\iffoo creates \footrue, \foofalse to go with \iffoo.
\outer\def\newif#1{\count@\escapechar \escapechar\m@ne
  \expandafter\expandafter\expandafter
   \def\@if#1{true}{\let#1=\iftrue}%
  \expandafter\expandafter\expandafter
   \def\@if#1{false}{\let#1=\iffalse}%
  \@if#1{false}\escapechar\count@} % the condition starts out false
\def\@if#1#2{\csname\expandafter\if@\string#1#2\endcsname}
{\uccode`1=`i \uccode`2=`f \uppercase{\gdef\if@12{}}} % `if' is required


\message{parameters,}

% Page size A4
\newdimen\pageheight     \pageheight=297mm
\newdimen\pagewidth      \pagewidth=210mm
\newdimen\headlineheight \headlineheight=20mm
\newdimen\footlineheight \footlineheight=25mm
\pdfpageheight=\pageheight
\pdfpagewidth=\pagewidth

\pretolerance=100
\tolerance=200 % INITEX sets this to 10000
\hbadness=1000
\vbadness=1000
\linepenalty=10
\hyphenpenalty=50
\exhyphenpenalty=50
\binoppenalty=700
\relpenalty=500
\clubpenalty=150
\widowpenalty=150
\displaywidowpenalty=50
\brokenpenalty=100
\predisplaypenalty=10000
% \postdisplaypenalty=0
% \interlinepenalty=0
% \floatingpenalty=0, set during \insert
% \outputpenalty=0, set before TeX enters \output
\doublehyphendemerits=10000
\finalhyphendemerits=5000
\adjdemerits=10000
% \looseness=0, cleared by TeX after each paragraph
% \pausing=0
% \holdinginserts=0
% \tracingonline=0
% \tracingmacros=0
% \tracingstats=0
% \tracingparagraphs=0
% \tracingpages=0
% \tracingoutput=0
\tracinglostchars=1
% \tracingcommands=0
% \tracingrestores=0
% \language=0
\uchyph=1
% \lefthyphenmin=2 \righthyphenmin=3 set below
% \globaldefs=0
% \maxdeadcycles=25 % INITEX does this
% \hangafter=1 % INITEX does this, also TeX after each paragraph
% \fam=0
% \mag=1000 % INITEX does this
% \escapechar=`\\ % INITEX does this
\defaulthyphenchar=`\-
\defaultskewchar=-1
% \endlinechar=`\^^M % INITEX does this
\newlinechar=-1
\delimiterfactor=901
% \time=now % TeX does this at beginning of job
% \day=now % TeX does this at beginning of job
% \month=now % TeX does this at beginning of job
% \year=now % TeX does this at beginning of job
\showboxbreadth=5
\showboxdepth=3
\errorcontextlines=5

\hfuzz=0.1pt
\vfuzz=0.1pt
\overfullrule=0pt
\hsize=160mm
\vsize=248mm
\maxdepth=4pt
\splitmaxdepth=\maxdimen
\boxmaxdepth=\maxdimen
% \lineskiplimit=0pt, changed by \normalbaselines
\delimitershortfall=5pt
\nulldelimiterspace=1.2pt
\scriptspace=0.5pt
% \mathsurround=0pt
% \predisplaysize=0pt, set before TeX enters $$
% \displaywidth=0pt, set before TeX enters $$
% \displayindent=0pt, set before TeX enters $$
\parindent=0pt
% \hangindent=0pt, zeroed by TeX after each paragraph
\hoffset=-0.4mm %+25.4mm = 25mm
\voffset=4.6mm %+25.4mm = 30mm

% \baselineskip=0pt, changed by \normalbaselines
% \lineskip=0pt, changed by \normalbaselines
\parskip=0pt plus 2pt
\abovedisplayskip=12pt plus 3pt minus 9pt
\abovedisplayshortskip=0pt plus 3pt
\belowdisplayskip=12pt plus 3pt minus 9pt
\belowdisplayshortskip=7pt plus 3pt minus 4pt
% \leftskip=0pt
% \rightskip=0pt
\topskip=10pt
\splittopskip=10pt
% \tabskip=0pt
% \spaceskip=0pt
% \xspaceskip=0pt
\parfillskip=0pt plus 1fil

\thinmuskip=3mu
\medmuskip=4mu plus 2mu minus 4mu
\thickmuskip=5mu plus 5mu

% We also define special registers that function like parameters:
\newskip\smallskipamount \smallskipamount=3pt plus 1pt minus 1pt
\newskip\medskipamount \medskipamount=6pt plus 2pt minus 2pt
\newskip\bigskipamount \bigskipamount=12pt plus 4pt minus 4pt
\newskip\normalbaselineskip \normalbaselineskip=12pt
\newskip\normallineskip \normallineskip=1pt
\newdimen\normallineskiplimit \normallineskiplimit=0pt
\newdimen\jot \jot=3pt
\newcount\interdisplaylinepenalty \interdisplaylinepenalty=100
\newcount\interfootnotelinepenalty \interfootnotelinepenalty=100

% Definitions for preloaded fonts

\def\magstephalf{1095 }
\def\magstep#1{\ifcase#1 \@m\or 1200\or 1440\or 1728\or 2074\or 2488\fi\relax}

\message{fonts,}

\def\@font#1#2{
\global\expandafter\font\csname #1\endcsname
= #2}


\def\edotex@newfont#1#2#3#4{%1name%2fontname%3family%4short
\expandafter\@font{ten#1}{#2 at10pt} 
\expandafter\@font{nine#1}{#2 at9pt}
\expandafter\@font{seven#1}{#2 at7pt} 
\expandafter\@font{five#1}{#2 at5pt} 
\expandafter\@font{twelv#1}{#2 at12pt}
\expandafter\@font{twenty#1}{#2 at20pt} 

\global\textfont#3=\csname ten#1\endcsname
\global\scriptfont#3=\csname seven#1\endcsname
\global\scriptscriptfont#3=\csname five#1\endcsname
\expandafter\gdef\csname #4\endcsname{\fam=#3 \csname ten#1\endcsname}
\expandafter\gdef\csname #4small\endcsname{\fam=#3 \csname nine#1\endcsname}
\expandafter\gdef\csname #4large\endcsname{\fam=#3 \csname twelv#1\endcsname}
\expandafter\gdef\csname #4big\endcsname{\fam=#3 \csname twenty#1\endcsname}
}

\let\newfont\edotex@newfont


% Family 0 (Roman/Sans)
\newfont{rm}{phvr8r}{0}{rm}
% Family 1 (Math italic)
\newfont{i}{cmmi10}{1}{mit}
% Family 2 (Math symbols)
\newfont{sy}{cmsy10}{2}{cal}
% Family 3 (Math extension)
\newfont{ex}{cmex10}{3}{ex}
% Family 4 (Italic text)
\newfont{it}{phvro8r}{4}{it}
% Family 5 (Slanted text)
\newfont{sl}{phvro8r}{5}{sl}
% Family 6 (Bold text)
\newfont{bf}{phvb8r}{6}{bf}
% Family 7 (Typewriter)
\newfont{tt}{phvr8r}{7}{tt}
% Family 8 (Roman/Sans Small Caps)
\newfont{sc}{phvr8r}{8}{sc}
% Family 9 (Bold text Small Caps)
\newfont{bfsc}{phvb8r}{8}{bfsc}
\rm % Sets normal roman font


\message{macros,} 

\def\frenchspacing{\sfcode`\.\@m \sfcode`\?\@m \sfcode`\!\@m
  \sfcode`\:\@m \sfcode`\;\@m \sfcode`\,\@m}
\def\nonfrenchspacing{\sfcode`\.3000\sfcode`\?3000\sfcode`\!3000%
  \sfcode`\:2000\sfcode`\;1500\sfcode`\,1250 }

\def\normalbaselines{\lineskip\normallineskip
  \baselineskip\normalbaselineskip \lineskiplimit\normallineskiplimit}

\def\^^M{\ } % control <return> = control <space>
\def\^^I{\ } % same for <tab>

\def\lq{`} \def\rq{'}
\def\lbrack{[} \def\rbrack{]}

\let\endgraf=\par \let\endline=\cr

\def\space{ }
\def\empty{}
\def\null{\hbox{}}

\let\bgroup={ \let\egroup=}

% In \obeylines, we say `\let^^M=\par' instead of `\def^^M{\par}'
% since this allows, for example, `\let\par=\cr \obeylines \halign{...'
{\catcode`\^^M=\active % these lines must end with %
  \gdef\obeylines{\catcode`\^^M\active \let^^M\par}%
  \global\let^^M\par} % this is in case ^^M appears in a \write
\def\obeyspaces{\catcode`\ \active}
{\obeyspaces\global\let =\space}

\def\loop#1\repeat{\def\body{#1}\iterate}
\def\iterate{\body \let\next\iterate \else\let\next\relax\fi \next}
\let\repeat=\fi % this makes \loop...\if...\repeat skippable

\def\thinspace{\kern .16667em }
\def\negthinspace{\kern-.16667em }
\def\enspace{\kern.5em }

\def\enskip{\hskip.5em\relax}
\def\quad{\hskip1em\relax}
\def\qquad{\hskip2em\relax}

\def\smallskip{\vskip\smallskipamount}
\def\medskip{\vskip\medskipamount}
\def\bigskip{\vskip\bigskipamount}

\def\nointerlineskip{\prevdepth-1000\p@}
\def\offinterlineskip{\baselineskip-1000\p@
  \lineskip\z@ \lineskiplimit\maxdimen}

\def\topglue{\nointerlineskip\vglue-\topskip\vglue} % for top of page
\def\vglue{\afterassignment\vgl@\skip@=}
\def\vgl@{\par \dimen@\prevdepth \hrule height\z@
  \nobreak\vskip\skip@ \prevdepth\dimen@}
\def\hglue{\afterassignment\hgl@\skip@=}
\def\hgl@{\leavevmode \count@\spacefactor \vrule width\z@
  \nobreak\hskip\skip@ \spacefactor\count@}

\def~{\penalty\@M \ } % tie
\def\slash{/\penalty\exhyphenpenalty} % a `/' that acts like a `-'

\def\break{\penalty-\@M}
\def\nobreak{\penalty \@M}
\def\allowbreak{\penalty \z@}

\def\filbreak{\par\vfil\penalty-200\vfilneg}
\def\goodbreak{\par\penalty-500 }
\def\eject{\par\break}
\def\supereject{\par\penalty-\@MM}

\def\removelastskip{\ifdim\lastskip=\z@\else\vskip-\lastskip\fi}
\def\smallbreak{\par\ifdim\lastskip<\smallskipamount
  \removelastskip\penalty-50\smallskip\fi}
\def\medbreak{\par\ifdim\lastskip<\medskipamount
  \removelastskip\penalty-100\medskip\fi}
\def\bigbreak{\par\ifdim\lastskip<\bigskipamount
  \removelastskip\penalty-200\bigskip\fi}

\def\line{\hbox to\hsize}
\def\leftline#1{\line{#1\hss}}
\def\rightline#1{\line{\hss#1}}
\def\centerline#1{\line{\hss#1\hss}}

\def\rlap#1{\hbox to\z@{#1\hss}}
\def\llap#1{\hbox to\z@{\hss#1}}

\def\m@th{\mathsurround\z@}
\def\underbar#1{$\setbox\z@\hbox{#1}\dp\z@\z@
  \m@th \underline{\box\z@}$}

\newbox\strutbox
\setbox\strutbox=\hbox{\vrule height8.5pt depth3.5pt width\z@}
\def\strut{\relax\ifmmode\copy\strutbox\else\unhcopy\strutbox\fi}

\def\hidewidth{\hskip\hideskip} % for alignment entries that can stick out
\def\ialign{\everycr{}\tabskip\z@skip\halign} % initialized \halign
\newcount\mscount
\def\multispan#1{\omit \mscount#1\relax
  \loop\ifnum\mscount>\@ne \sp@n\repeat}
\def\sp@n{\span\omit\advance\mscount\m@ne}

\newif\ifus@ \newif\if@cr
\newbox\tabs \newbox\tabsyet \newbox\tabsdone

\def\cleartabs{\global\setbox\tabsyet\null \setbox\tabs\null}
\def\settabs{\setbox\tabs\null \futurelet\next\sett@b}
\let\+=\relax % in case this file is being read in twice
\def\sett@b{\ifx\next\+\def\nxt{\afterassignment\s@tt@b\let\nxt}%
  \else\let\nxt\s@tcols\fi \let\next\relax \nxt}
\def\s@tt@b{\let\nxt\relax \us@false\m@ketabbox}
\def\tabalign{\us@true\m@ketabbox} % non-\outer version of \+
\outer\def\+{\tabalign}
\def\s@tcols#1\columns{\count@#1\dimen@\hsize
  \loop\ifnum\count@>\z@ \@nother \repeat}
\def\@nother{\dimen@ii\dimen@ \divide\dimen@ii\count@
  \setbox\tabs\hbox{\hbox to\dimen@ii{}\unhbox\tabs}%
  \advance\dimen@-\dimen@ii \advance\count@\m@ne}

\def\m@ketabbox{\begingroup
  \global\setbox\tabsyet\copy\tabs
  \global\setbox\tabsdone\null
  \def\cr{\@crtrue\crcr\egroup\egroup
    \ifus@\unvbox\z@\lastbox\fi\endgroup
    \setbox\tabs\hbox{\unhbox\tabsyet\unhbox\tabsdone}}%
  \setbox\z@\vbox\bgroup\@crfalse
    \ialign\bgroup&\t@bbox##\t@bb@x\crcr}

\def\t@bbox{\setbox\z@\hbox\bgroup}
\def\t@bb@x{\if@cr\egroup % now \box\z@ holds the column
  \else\hss\egroup \global\setbox\tabsyet\hbox{\unhbox\tabsyet
      \global\setbox\@ne\lastbox}% now \box\@ne holds its size
    \ifvoid\@ne\global\setbox\@ne\hbox to\wd\z@{}%
    \else\setbox\z@\hbox to\wd\@ne{\unhbox\z@}\fi
    \global\setbox\tabsdone\hbox{\box\@ne\unhbox\tabsdone}\fi
  \box\z@}

\def\hang{\hangindent\parindent}
\def\textindent#1{\indent\llap{#1\enspace}\ignorespaces}
\def\item{\par\hang\textindent}
\def\itemitem{\par\indent \hangindent2\parindent \textindent}
\def\narrower{\advance\leftskip\parindent
  \advance\rightskip\parindent}

\outer\def\beginsection#1\par{\vskip\z@ plus.3\vsize\penalty-250
  \vskip\z@ plus-.3\vsize\bigskip\vskip\parskip
  \message{#1}\leftline{\bf#1}\nobreak\smallskip\noindent}
\outer\def\proclaim #1. #2\par{\medbreak
  \noindent{\bf#1.\enspace}{\sl#2\par}%
  \ifdim\lastskip<\medskipamount \removelastskip\penalty55\medskip\fi}

\def\raggedright{\rightskip\z@ plus2em \spaceskip.3333em \xspaceskip.5em\relax}
\def\ttraggedright{\tt\rightskip\z@ plus2em\relax} % for use with \tt only

\chardef\%=`\%
\chardef\&=`\&
\chardef\#=`\#
\chardef\$=`\$
\chardef\ss="19
\chardef\ae="1A
\chardef\oe="1B
\chardef\o="1C
\chardef\AE="1D
\chardef\OE="1E
\chardef\O="1F
\chardef\i="10 \chardef\j="11 % dotless letters
\def\aa{\accent23a}
\def\l{\char32l}
\def\L{\leavevmode\setbox0\hbox{L}\hbox to\wd0{\hss\char32L}}

\def\leavevmode{\unhbox\voidb@x} % begins a paragraph, if necessary
\def\_{\leavevmode \kern.06em \vbox{\hrule width.3em}}
\def\AA{\leavevmode\setbox0\hbox{!}\dimen@\ht0\advance\dimen@-1ex%
  \rlap{\raise.67\dimen@\hbox{\char'27}}A}

\def\mathhexbox#1#2#3{\leavevmode
  \hbox{$\m@th \mathchar"#1#2#3$}}
\def\dag{\mathhexbox279}
\def\ddag{\mathhexbox27A}
\def\S{\mathhexbox278}
\def\P{\mathhexbox27B}
\def\Orb{\mathhexbox20D}

\def\oalign#1{\leavevmode\vtop{\baselineskip\z@skip \lineskip.25ex%
  \ialign{##\crcr#1\crcr}}} \def\o@lign{\lineskiplimit\z@ \oalign}
\def\ooalign{\lineskiplimit-\maxdimen \oalign} % chars over each other
{\catcode`p=12 \catcode`t=12 \gdef\\#1pt{#1}} \let\getf@ctor=\\
\def\sh@ft#1{\dimen@#1\kern\expandafter\getf@ctor\the\fontdimen1\font
  \dimen@} % kern by #1 times the current slant
\def\d#1{{\o@lign{\relax#1\crcr\hidewidth\sh@ft{-1ex}.\hidewidth}}}
\def\b#1{{\o@lign{\relax#1\crcr\hidewidth\sh@ft{-3ex}%
    \vbox to.2ex{\hbox{\char22}\vss}\hidewidth}}}
\def\c#1{{\setbox\z@\hbox{#1}\ifdim\ht\z@=1ex\accent24 #1%
  \else\ooalign{\unhbox\z@\crcr\hidewidth\char24\hidewidth}\fi}}
\def\copyright{{\ooalign{\hfil\raise.07ex\hbox{c}\hfil\crcr\Orb}}}

\def\dots{\relax\ifmmode\ldots\else$\m@th\ldots\,$\fi}
\def\TeX{T\kern-.1667em\lower.5ex\hbox{E}\kern-.125emX}
\def\eduTeX{eduT\kern-.1667em\lower.5ex\hbox{E}\kern-.125emX}

\def\`#1{{\accent18 #1}}
\def\'#1{{\accent19 #1}}
\def\v#1{{\accent20 #1}} \let\^^_=\v
\def\u#1{{\accent21 #1}} \let\^^S=\u
\def\=#1{{\accent22 #1}}
\def\^#1{{\accent94 #1}} \let\^^D=\^
\def\.#1{{\accent95 #1}}
\def\H#1{{\accent"7D #1}}
\def\~#1{{\accent"7E #1}}
\def\"#1{{\accent"7F #1}}
\def\t#1{{\edef\next{\the\font}\the\textfont1\accent"7F\next#1}}

\def\hrulefill{\leaders\hrule\hfill}
\def\dotfill{\cleaders\hbox{$\m@th \mkern1.5mu.\mkern1.5mu$}\hfill}
\def\rightarrowfill{$\m@th\smash-\mkern-7mu%
  \cleaders\hbox{$\mkern-2mu\smash-\mkern-2mu$}\hfill
  \mkern-7mu\mathord\rightarrow$}
\def\leftarrowfill{$\m@th\mathord\leftarrow\mkern-7mu%
  \cleaders\hbox{$\mkern-2mu\smash-\mkern-2mu$}\hfill
  \mkern-7mu\smash-$}
\mathchardef\braceld="37A \mathchardef\bracerd="37B
\mathchardef\bracelu="37C \mathchardef\braceru="37D
\def\downbracefill{$\m@th \setbox\z@\hbox{$\braceld$}%
  \braceld\leaders\vrule height\ht\z@ depth\z@\hfill\braceru
  \bracelu\leaders\vrule height\ht\z@ depth\z@\hfill\bracerd$}
\def\upbracefill{$\m@th \setbox\z@\hbox{$\braceld$}%
  \bracelu\leaders\vrule height\ht\z@ depth\z@\hfill\bracerd
  \braceld\leaders\vrule height\ht\z@ depth\z@\hfill\braceru$}

\outer\def\bye{\par\vfill\supereject\end}
%\outer\def\end{\vfill\eject\csname bye\expandafter\endcsname}

%Link
\def\weblink(#1)#2{\hbox{\pdfstartlink user{/Subtype /Link /A << /Type /Action /S /URI/URI (#1) >>} #2\pdfendlink}}
\def\link(#1)#2{\hbox{\pdfstartlink goto page #1 {/Fit}  #2\pdfendlink}}

%Colors
\chardef\@ColorStack=0%\pdfcolorstackinit page direct{0 g 0 G}
%  define colors [rgb]
\def\defineColor(#1,#2,#3)#4{\expandafter\xdef\csname @ColorName@#4\endcsname##1{\pdfcolorstack \@ColorStack push {#1 #2 #3 rg} ##1 \pdfcolorstack \@ColorStack pop}}
\def\useColor#1{\csname @ColorName@#1\endcsname}

% Include Images
\def\includeImage[#1]#2{
\pdfximage width #1 {#2} \pdfrefximage\pdflastximage}


\def\scaleBox[#1]#2{\setbox0=\hbox{#2}
\pdfsave\pdfsetmatrix{#1 0 0 #1}\rlap{\smash{\copy0}}\pdfrestore
{\setbox2=\hbox{}\wd2=#1\wd0 \ht2=#1\ht0 \dp2=#1\dp0 \box2}
}


\def\PDF#1{\special{pdf:#1}}
\def\PDFD#1{\special{pdf:direct:#1}}
\def\PDF@Gpush{\PDF{q}}
\def\PDF@Gpop{\PDF{Q}}

% Latex
\def\makeatletter{\catcode`@=11}
\def\makeatother{\catcode`@=12}


% Additional macros
% new page
\def\newpage{\vfil\eject} 
% new line
\def\newline{\hfil\break}

% new variable
\def\newvar#1{\expandafter\xdef\csname #1\endcsname##1{\xdef\csname the#1\endcsname{##1}}\csname #1\endcsname{#1}} 


% convert to interger to letter [a-z]
\def\alphanumeral#1{\ifcase#1 ?\or a\or b\or c\or d\or e\or f\or g\or h\or i\or j\or k\or l\or m\or n\or o\or p\or q\or r\or s\or t\or u\or v\or w\or x\or y\or z\else xx\fi}
\def\monthName#1{\ifcase#1\or January\or February\or March\or April\or May\or June\or July\or August\or September\or October\or November\or December\fi}

%if star is added
\def\@ifstar#1#2{\let\@wstar#1 \let\@nstar#2 \futurelet\@next@char\@ifstarh}
\def\@ifstarh{\ifx\@next@char *\expandafter\@wstar\else\expandafter\@nstar\fi}


%if optional argument is given
\def\@optarg#1#2{\let\@woptarg#1 \let\@noptarg#2 \futurelet\@next@char\@optargh}
\def\@optargh{\ifx\@next@char [ \expandafter\@woptarg \else\expandafter\@noptarg\fi}

%remove pt from dimension
\edef\@rmpt{\def\noexpand\@rmpt##1.##2\string p\string t{##1.##2}} \@rmpt
\def\@strip@pt{\expandafter\@rmpt\the}

%remove pt from dimension and cut decimal
\edef\@intrmpt{\def\noexpand\@intrmpt##1.##2\string p\string t{##1}} \@intrmpt
\def\@dimtoint{\expandafter\@intrmpt\the}

\def\draft{\overfullrule=5pt}

\message{variables} %added
% Locales
\def\@worksheetName{Worksheet}
\def\@exerciseName{Exercise}
\def\@examName{Exam}
\def\@testName{Test}
\def\@pointsName{Points}
\def\@gradeName{Grade}
\def\@totalName{Total}
\def\@solutionName{Solution}
\def\@countName{Count}
\def\@wishName{Good Luck!}
\def\@nameName{Name}


\def\edo@LocalDate#1#2#3{\eduTex@Month{#2} #3,\ #1}

% Colors
\defineColor(0.6,0.6,0.6){Header}
\defineColor(0.6,0.6,0.6){Footer}
\defineColor(0.76,0.07,0.08){Highlight1}
\defineColor(0.42,0.42,0.42){Highlight2}
\defineColor(0.42,0.42,0.42){TitleColor}

\def\@topic{\edoTeX}
\def\@subtopic{Education \TeX\space Typesetting}
\def\@icon{\edoTeX}
\def\@initial{E}
\def\@url{www.benjaminfalkner.de}
\def\@author{Benjamin "Benno" Falkner}
\def\@title{\edoTeX}
\def\@subtitle{Education with \TeX\space Typsetting}
\def\@institute{\edoTeX\space school }
\def\@subject{\edoTeX}
\def\@date{\edoTeX}


\message{math definitions,}

\let\sp=^ \let\sb=_
\def\,{\mskip\thinmuskip}
\def\>{\mskip\medmuskip}
\def\;{\mskip\thickmuskip}
\def\!{\mskip-\thinmuskip}
\def\*{\discretionary{\thinspace\the\textfont2\char2}{}{}}
{\catcode`\'=\active \gdef'{^\bgroup\prim@s}}
\def\prim@s{\prime\futurelet\next\pr@m@s}
\def\pr@m@s{\ifx'\next\let\nxt\pr@@@s \else\ifx^\next\let\nxt\pr@@@t
  \else\let\nxt\egroup\fi\fi \nxt}
\def\pr@@@s#1{\prim@s} \def\pr@@@t#1#2{#2\egroup}
{\catcode`\^^Z=\active \gdef^^Z{\not=}} % ^^Z is like \ne in math

{\catcode`\_=\active \global\let_=\_} % _ in math is either subscript or \_

\mathchardef\alpha="010B
\mathchardef\beta="010C
\mathchardef\gamma="010D
\mathchardef\delta="010E
\mathchardef\epsilon="010F
\mathchardef\zeta="0110
\mathchardef\eta="0111
\mathchardef\theta="0112
\mathchardef\iota="0113
\mathchardef\kappa="0114
\mathchardef\lambda="0115
\mathchardef\mu="0116
\mathchardef\nu="0117
\mathchardef\xi="0118
\mathchardef\pi="0119
\mathchardef\rho="011A
\mathchardef\sigma="011B
\mathchardef\tau="011C
\mathchardef\upsilon="011D
\mathchardef\phi="011E
\mathchardef\chi="011F
\mathchardef\psi="0120
\mathchardef\omega="0121
\mathchardef\varepsilon="0122
\mathchardef\vartheta="0123
\mathchardef\varpi="0124
\mathchardef\varrho="0125
\mathchardef\varsigma="0126
\mathchardef\varphi="0127
\mathchardef\Gamma="7000
\mathchardef\Delta="7001
\mathchardef\Theta="7002
\mathchardef\Lambda="7003
\mathchardef\Xi="7004
\mathchardef\Pi="7005
\mathchardef\Sigma="7006
\mathchardef\Upsilon="7007
\mathchardef\Phi="7008
\mathchardef\Psi="7009
\mathchardef\Omega="700A

\mathchardef\aleph="0240
\def\hbar{{\mathchar'26\mkern-9muh}}
\mathchardef\imath="017B
\mathchardef\jmath="017C
\mathchardef\ell="0160
\mathchardef\wp="017D
\mathchardef\Re="023C
\mathchardef\Im="023D
\mathchardef\partial="0140
\mathchardef\infty="0231
\mathchardef\prime="0230
\mathchardef\emptyset="023B
\mathchardef\nabla="0272
\def\surd{{\mathchar"1270}}
\mathchardef\top="023E
\mathchardef\bot="023F
\def\angle{{\vbox{\ialign{$\m@th\scriptstyle##$\crcr
      \not\mathrel{\mkern14mu}\crcr
      \noalign{\nointerlineskip}
      \mkern2.5mu\leaders\hrule height.34pt\hfill\mkern2.5mu\crcr}}}}
\mathchardef\triangle="0234
\mathchardef\forall="0238
\mathchardef\exists="0239
\mathchardef\neg="023A \let\lnot=\neg
\mathchardef\flat="015B
\mathchardef\natural="015C
\mathchardef\sharp="015D
\mathchardef\clubsuit="027C
\mathchardef\diamondsuit="027D
\mathchardef\heartsuit="027E
\mathchardef\spadesuit="027F

\mathchardef\coprod="1360
\mathchardef\bigvee="1357
\mathchardef\bigwedge="1356
\mathchardef\biguplus="1355
\mathchardef\bigcap="1354
\mathchardef\bigcup="1353
\mathchardef\intop="1352 \def\int{\intop\nolimits}
\mathchardef\prod="1351
\mathchardef\sum="1350
\mathchardef\bigotimes="134E
\mathchardef\bigoplus="134C
\mathchardef\bigodot="134A
\mathchardef\ointop="1348 \def\oint{\ointop\nolimits}
\mathchardef\bigsqcup="1346
\mathchardef\smallint="1273

\mathchardef\triangleleft="212F
\mathchardef\triangleright="212E
\mathchardef\bigtriangleup="2234
\mathchardef\bigtriangledown="2235
\mathchardef\wedge="225E \let\land=\wedge
\mathchardef\vee="225F \let\lor=\vee
\mathchardef\cap="225C
\mathchardef\cup="225B
\mathchardef\ddagger="227A
\mathchardef\dagger="2279
\mathchardef\sqcap="2275
\mathchardef\sqcup="2274
\mathchardef\uplus="225D
\mathchardef\amalg="2271
\mathchardef\diamond="2205
\mathchardef\bullet="220F
\mathchardef\wr="226F
\mathchardef\div="2204
\mathchardef\odot="220C
\mathchardef\oslash="220B
\mathchardef\otimes="220A
\mathchardef\ominus="2209
\mathchardef\oplus="2208
\mathchardef\mp="2207
\mathchardef\pm="2206
\mathchardef\circ="220E
\mathchardef\bigcirc="220D
\mathchardef\setminus="226E % for set difference A\setminus B
\mathchardef\cdot="2201
\mathchardef\ast="2203
\mathchardef\times="2202
\mathchardef\star="213F

\mathchardef\propto="322F
\mathchardef\sqsubseteq="3276
\mathchardef\sqsupseteq="3277
\mathchardef\parallel="326B
\mathchardef\mid="326A
\mathchardef\dashv="3261
\mathchardef\vdash="3260
\mathchardef\nearrow="3225
\mathchardef\searrow="3226
\mathchardef\nwarrow="322D
\mathchardef\swarrow="322E
\mathchardef\Leftrightarrow="322C
\mathchardef\Leftarrow="3228
\mathchardef\Rightarrow="3229
\def\neq{\not=} \let\ne=\neq
\mathchardef\leq="3214 \let\le=\leq
\mathchardef\geq="3215 \let\ge=\geq
\mathchardef\succ="321F
\mathchardef\prec="321E
\mathchardef\approx="3219
\mathchardef\succeq="3217
\mathchardef\preceq="3216
\mathchardef\supset="321B
\mathchardef\subset="321A
\mathchardef\supseteq="3213
\mathchardef\subseteq="3212
\mathchardef\in="3232
\mathchardef\ni="3233 \let\owns=\ni
\mathchardef\gg="321D
\mathchardef\ll="321C
\mathchardef\not="3236
\mathchardef\leftrightarrow="3224
\mathchardef\leftarrow="3220 \let\gets=\leftarrow
\mathchardef\rightarrow="3221 \let\to=\rightarrow
\mathchardef\mapstochar="3237 \def\mapsto{\mapstochar\rightarrow}
\mathchardef\sim="3218
\mathchardef\simeq="3227
\mathchardef\perp="323F
\mathchardef\equiv="3211
\mathchardef\asymp="3210
\mathchardef\smile="315E
\mathchardef\frown="315F
\mathchardef\leftharpoonup="3128
\mathchardef\leftharpoondown="3129
\mathchardef\rightharpoonup="312A
\mathchardef\rightharpoondown="312B

\def\joinrel{\mathrel{\mkern-3mu}}
\def\relbar{\mathrel{\smash-}} % \smash, because - has the same height as +
\def\Relbar{\mathrel=}
\mathchardef\lhook="312C \def\hookrightarrow{\lhook\joinrel\rightarrow}
\mathchardef\rhook="312D \def\hookleftarrow{\leftarrow\joinrel\rhook}
\def\bowtie{\mathrel\triangleright\joinrel\mathrel\triangleleft}
\def\models{\mathrel|\joinrel=}
\def\Longrightarrow{\Relbar\joinrel\Rightarrow}
\def\longrightarrow{\relbar\joinrel\rightarrow}
\def\longleftarrow{\leftarrow\joinrel\relbar}
\def\Longleftarrow{\Leftarrow\joinrel\Relbar}
\def\longmapsto{\mapstochar\longrightarrow}
\def\longleftrightarrow{\leftarrow\joinrel\rightarrow}
\def\Longleftrightarrow{\Leftarrow\joinrel\Rightarrow}
\def\iff{\;\Longleftrightarrow\;}

\mathchardef\ldotp="613A % ldot as a punctuation mark
\mathchardef\cdotp="6201 % cdot as a punctuation mark
\mathchardef\colon="603A % colon as a punctuation mark
\def\ldots{\mathinner{\ldotp\ldotp\ldotp}}
\def\cdots{\mathinner{\cdotp\cdotp\cdotp}}
\def\vdots{\vbox{\baselineskip4\p@ \lineskiplimit\z@
    \kern6\p@\hbox{.}\hbox{.}\hbox{.}}}
\def\ddots{\mathinner{\mkern1mu\raise7\p@\vbox{\kern7\p@\hbox{.}}\mkern2mu
    \raise4\p@\hbox{.}\mkern2mu\raise\p@\hbox{.}\mkern1mu}}

\def\acute{\mathaccent"7013 }
\def\grave{\mathaccent"7012 }
\def\ddot{\mathaccent"707F }
\def\tilde{\mathaccent"707E }
\def\bar{\mathaccent"7016 }
\def\breve{\mathaccent"7015 }
\def\check{\mathaccent"7014 }
\def\hat{\mathaccent"705E }
\def\vec{\mathaccent"017E }
\def\dot{\mathaccent"705F }
\def\widetilde{\mathaccent"0365 }
\def\widehat{\mathaccent"0362 }
\def\overrightarrow#1{\vbox{\m@th\ialign{##\crcr
      \rightarrowfill\crcr\noalign{\kern-\p@\nointerlineskip}
      $\hfil\displaystyle{#1}\hfil$\crcr}}}
\def\overleftarrow#1{\vbox{\m@th\ialign{##\crcr
      \leftarrowfill\crcr\noalign{\kern-\p@\nointerlineskip}
      $\hfil\displaystyle{#1}\hfil$\crcr}}}
\def\overbrace#1{\mathop{\vbox{\m@th\ialign{##\crcr\noalign{\kern3\p@}
      \downbracefill\crcr\noalign{\kern3\p@\nointerlineskip}
      $\hfil\displaystyle{#1}\hfil$\crcr}}}\limits}
\def\underbrace#1{\mathop{\vtop{\m@th\ialign{##\crcr
      $\hfil\displaystyle{#1}\hfil$\crcr\noalign{\kern3\p@\nointerlineskip}
      \upbracefill\crcr\noalign{\kern3\p@}}}}\limits}
\def\skew#1#2#3{{\muskip\z@#1mu\divide\muskip\z@\tw@ \mkern\muskip\z@
    #2{\mkern-\muskip\z@{#3}\mkern\muskip\z@}\mkern-\muskip\z@}{}}

\def\lmoustache{\delimiter"437A340 } % top from (, bottom from )
\def\rmoustache{\delimiter"537B341 } % top from ), bottom from (
\def\lgroup{\delimiter"462833A } % extensible ( with sharper tips
\def\rgroup{\delimiter"562933B } % extensible ) with sharper tips
\def\arrowvert{\delimiter"26A33C } % arrow without arrowheads
\def\Arrowvert{\delimiter"26B33D } % double arrow without arrowheads
\def\bracevert{\delimiter"77C33E } % the vertical bar that extends braces
\def\Vert{\delimiter"26B30D } \let\|=\Vert
\def\vert{\delimiter"26A30C }
\def\uparrow{\delimiter"3222378 }
\def\downarrow{\delimiter"3223379 }
\def\updownarrow{\delimiter"326C33F }
\def\Uparrow{\delimiter"322A37E }
\def\Downarrow{\delimiter"322B37F }
\def\Updownarrow{\delimiter"326D377 }
\def\backslash{\delimiter"26E30F } % for double coset G\backslash H
\def\rangle{\delimiter"526930B }
\def\langle{\delimiter"426830A }
\def\rbrace{\delimiter"5267309 } \let\}=\rbrace
\def\lbrace{\delimiter"4266308 } \let\{=\lbrace
\def\rceil{\delimiter"5265307 }
\def\lceil{\delimiter"4264306 }
\def\rfloor{\delimiter"5263305 }
\def\lfloor{\delimiter"4262304 }

\def\bigl{\mathopen\big}
\def\bigm{\mathrel\big}
\def\bigr{\mathclose\big}
\def\Bigl{\mathopen\Big}
\def\Bigm{\mathrel\Big}
\def\Bigr{\mathclose\Big}
\def\biggl{\mathopen\bigg}
\def\biggm{\mathrel\bigg}
\def\biggr{\mathclose\bigg}
\def\Biggl{\mathopen\Bigg}
\def\Biggm{\mathrel\Bigg}
\def\Biggr{\mathclose\Bigg}
\def\big#1{{\hbox{$\left#1\vbox to8.5\p@{}\right.\n@space$}}}
\def\Big#1{{\hbox{$\left#1\vbox to11.5\p@{}\right.\n@space$}}}
\def\bigg#1{{\hbox{$\left#1\vbox to14.5\p@{}\right.\n@space$}}}
\def\Bigg#1{{\hbox{$\left#1\vbox to17.5\p@{}\right.\n@space$}}}
\def\n@space{\nulldelimiterspace\z@ \m@th}

\def\choose{\atopwithdelims()}
\def\brack{\atopwithdelims[]}
\def\brace{\atopwithdelims\{\}}

\def\sqrt{\radical"270370 }

\def\mathpalette#1#2{\mathchoice{#1\displaystyle{#2}}%
  {#1\textstyle{#2}}{#1\scriptstyle{#2}}{#1\scriptscriptstyle{#2}}}
\newbox\rootbox
\def\root#1\of{\setbox\rootbox
  \hbox{$\m@th\scriptscriptstyle{#1}$}\mathpalette\r@@t}
\def\r@@t#1#2{\setbox\z@\hbox{$\m@th#1\sqrt{#2}$}\dimen@\ht\z@
  \advance\dimen@-\dp\z@
  \mkern5mu\raise.6\dimen@\copy\rootbox \mkern-10mu\box\z@}
\newif\ifv@ \newif\ifh@
\def\vphantom{\v@true\h@false\ph@nt}
\def\hphantom{\v@false\h@true\ph@nt}
\def\phantom{\v@true\h@true\ph@nt}
\def\ph@nt{\ifmmode\def\next{\mathpalette\mathph@nt}%
  \else\let\next\makeph@nt\fi\next}
\def\makeph@nt#1{\setbox\z@\hbox{#1}\finph@nt}
\def\mathph@nt#1#2{\setbox\z@\hbox{$\m@th#1{#2}$}\finph@nt}
\def\finph@nt{\setbox\tw@\null
  \ifv@ \ht\tw@\ht\z@ \dp\tw@\dp\z@\fi
  \ifh@ \wd\tw@\wd\z@\fi \box\tw@}
\def\mathstrut{\vphantom(}
\def\smash{\relax % \relax, in case this comes first in \halign
  \ifmmode\def\next{\mathpalette\mathsm@sh}\else\let\next\makesm@sh
  \fi\next}
\def\makesm@sh#1{\setbox\z@\hbox{#1}\finsm@sh}
\def\mathsm@sh#1#2{\setbox\z@\hbox{$\m@th#1{#2}$}\finsm@sh}
\def\finsm@sh{\ht\z@\z@ \dp\z@\z@ \box\z@}

\def\cong{\mathrel{\mathpalette\@vereq\sim}} % congruence sign
\def\@vereq#1#2{\lower.5\p@\vbox{\lineskiplimit\maxdimen\lineskip-.5\p@
    \ialign{$\m@th#1\hfil##\hfil$\crcr#2\crcr=\crcr}}}
\def\notin{\mathrel{\mathpalette\c@ncel\in}}
\def\c@ncel#1#2{\m@th\ooalign{$\hfil#1\mkern1mu/\hfil$\crcr$#1#2$}}
\def\rightleftharpoons{\mathrel{\mathpalette\rlh@{}}}
\def\rlh@#1{\vcenter{\m@th\hbox{\ooalign{\raise2pt
          \hbox{$#1\rightharpoonup$}\crcr
        $#1\leftharpoondown$}}}}
\def\buildrel#1\over#2{\mathrel{\mathop{\kern\z@#2}\limits^{#1}}}
\def\doteq{\buildrel\textstyle.\over=}

\def\log{\mathop{\rm log}\nolimits}
\def\lg{\mathop{\rm lg}\nolimits}
\def\ln{\mathop{\rm ln}\nolimits}
\def\lim{\mathop{\rm lim}}
\def\limsup{\mathop{\rm lim\,sup}}
\def\liminf{\mathop{\rm lim\,inf}}
\def\sin{\mathop{\rm sin}\nolimits}
\def\arcsin{\mathop{\rm arcsin}\nolimits}
\def\sinh{\mathop{\rm sinh}\nolimits}
\def\cos{\mathop{\rm cos}\nolimits}
\def\arccos{\mathop{\rm arccos}\nolimits}
\def\cosh{\mathop{\rm cosh}\nolimits}
\def\tan{\mathop{\rm tan}\nolimits}
\def\arctan{\mathop{\rm arctan}\nolimits}
\def\tanh{\mathop{\rm tanh}\nolimits}
\def\cot{\mathop{\rm cot}\nolimits}
\def\coth{\mathop{\rm coth}\nolimits}
\def\sec{\mathop{\rm sec}\nolimits}
\def\csc{\mathop{\rm csc}\nolimits}
\def\max{\mathop{\rm max}}
\def\min{\mathop{\rm min}}
\def\sup{\mathop{\rm sup}}
\def\inf{\mathop{\rm inf}}
\def\arg{\mathop{\rm arg}\nolimits}
\def\ker{\mathop{\rm ker}\nolimits}
\def\dim{\mathop{\rm dim}\nolimits}
\def\hom{\mathop{\rm hom}\nolimits}
\def\det{\mathop{\rm det}}
\def\exp{\mathop{\rm exp}\nolimits}
\def\Pr{\mathop{\rm Pr}}
\def\gcd{\mathop{\rm gcd}}
\def\deg{\mathop{\rm deg}\nolimits}

\def\bmod{\nonscript\mskip-\medmuskip\mkern5mu
  \mathbin{\rm mod}\penalty900\mkern5mu\nonscript\mskip-\medmuskip}
\def\pmod#1{\allowbreak\mkern18mu({\rm mod}\,\,#1)}

\def\cases#1{\left\{\,\vcenter{\normalbaselines\m@th
    \ialign{$##\hfil$&\quad##\hfil\crcr#1\crcr}}\right.}
\def\matrix#1{\null\,\vcenter{\normalbaselines\m@th
    \ialign{\hfil$##$\hfil&&\quad\hfil$##$\hfil\crcr
      \mathstrut\crcr\noalign{\kern-\baselineskip}
      #1\crcr\mathstrut\crcr\noalign{\kern-\baselineskip}}}\,}
\def\pmatrix#1{\left(\matrix{#1}\right)}
\newdimen\p@renwd
\setbox0=\hbox{\tenex B} \p@renwd=\wd0 % width of the big left (
\def\bordermatrix#1{\begingroup \m@th
  \setbox\z@\vbox{\def\cr{\crcr\noalign{\kern2\p@\global\let\cr\endline}}%
    \ialign{$##$\hfil\kern2\p@\kern\p@renwd&\thinspace\hfil$##$\hfil
      &&\quad\hfil$##$\hfil\crcr
      \omit\strut\hfil\crcr\noalign{\kern-\baselineskip}%
      #1\crcr\omit\strut\cr}}%
  \setbox\tw@\vbox{\unvcopy\z@\global\setbox\@ne\lastbox}%
  \setbox\tw@\hbox{\unhbox\@ne\unskip\global\setbox\@ne\lastbox}%
  \setbox\tw@\hbox{$\kern\wd\@ne\kern-\p@renwd\left(\kern-\wd\@ne
    \global\setbox\@ne\vbox{\box\@ne\kern2\p@}%
    \vcenter{\kern-\ht\@ne\unvbox\z@\kern-\baselineskip}\,\right)$}%
  \null\;\vbox{\kern\ht\@ne\box\tw@}\endgroup}

\def\openup{\afterassignment\@penup\dimen@=}
\def\@penup{\advance\lineskip\dimen@
  \advance\baselineskip\dimen@
  \advance\lineskiplimit\dimen@}
\def\eqalign#1{\null\,\vcenter{\openup\jot\m@th
  \ialign{\strut\hfil$\displaystyle{##}$&$\displaystyle{{}##}$\hfil
      \crcr#1\crcr}}\,}
\newif\ifdt@p
\def\displ@y{\global\dt@ptrue\openup\jot\m@th
  \everycr{\noalign{\ifdt@p \global\dt@pfalse \ifdim\prevdepth>-1000\p@
      \vskip-\lineskiplimit \vskip\normallineskiplimit \fi
      \else \penalty\interdisplaylinepenalty \fi}}}
\def\@lign{\tabskip\z@skip\everycr{}} % restore inside \displ@y
\def\displaylines#1{\displ@y \tabskip\z@skip
  \halign{\hbox to\displaywidth{$\@lign\hfil\displaystyle##\hfil$}\crcr
    #1\crcr}}
\def\eqalignno#1{\displ@y \tabskip\centering
  \halign to\displaywidth{\hfil$\@lign\displaystyle{##}$\tabskip\z@skip
    &$\@lign\displaystyle{{}##}$\hfil\tabskip\centering
    &\llap{$\@lign##$}\tabskip\z@skip\crcr
    #1\crcr}}
\def\leqalignno#1{\displ@y \tabskip\centering
  \halign to\displaywidth{\hfil$\@lign\displaystyle{##}$\tabskip\z@skip
    &$\@lign\displaystyle{{}##}$\hfil\tabskip\centering
    &\kern-\displaywidth\rlap{$\@lign##$}\tabskip\displaywidth\crcr
    #1\crcr}}



\message{output routines,} 
\newtoks\headline \headline={\hfil} % headline is normally blank
\newtoks\footline \footline={\hss\tenrm\folio\hss}
  % footline is normally a centered page number in font \tenrm
\newif\ifr@ggedbottom
\def\raggedbottom{\topskip 10\p@ plus60\p@ \r@ggedbottomtrue}
\def\normalbottom{\topskip 10\p@ \r@ggedbottomfalse} % undoes \raggedbottom
\def\folio{\ifnum\pageno<\z@ \romannumeral-\pageno \else\number\pageno \fi}
\def\nopagenumbers{\footline{\hfil}} % blank out the footline
\def\advancepageno{\ifnum\pageno<\z@ \global\advance\pageno\m@ne
  \else\global\advance\pageno\@ne \fi} % increase |pageno|

\newinsert\footins
\def\footnote#1{\let\@sf\empty % parameter #2 (the text) is read later
  \ifhmode\edef\@sf{\spacefactor\the\spacefactor}\/\fi
  #1\@sf\vfootnote{#1}}
\def\vfootnote#1{\insert\footins\bgroup
  \interlinepenalty\interfootnotelinepenalty
  \splittopskip\ht\strutbox % top baseline for broken footnotes
  \splitmaxdepth\dp\strutbox \floatingpenalty\@MM
  \leftskip\z@skip \rightskip\z@skip \spaceskip\z@skip \xspaceskip\z@skip
  \textindent{#1}\footstrut\futurelet\next\fo@t}
\def\fo@t{\ifcat\bgroup\noexpand\next \let\next\f@@t
  \else\let\next\f@t\fi \next}
\def\f@@t{\bgroup\aftergroup\@foot\let\next}
\def\f@t#1{#1\@foot}
\def\@foot{\strut\egroup}
\def\footstrut{\vbox to\splittopskip{}}
\skip\footins=\bigskipamount % space added when footnote is present
\count\footins=1000 % footnote magnification factor (1 to 1)
\dimen\footins=8in % maximum footnotes per page

\newinsert\topins
\newif\ifp@ge \newif\if@mid
\def\topinsert{\@midfalse\p@gefalse\@ins}
\def\midinsert{\@midtrue\@ins}
\def\pageinsert{\@midfalse\p@getrue\@ins}
\skip\topins=\z@skip % no space added when a topinsert is present
\count\topins=1000 % magnification factor (1 to 1)
\dimen\topins=\maxdimen % no limit per page
\def\@ins{\par\begingroup\setbox\z@\vbox\bgroup} % start a \vbox
\def\endinsert{\egroup % finish the \vbox
  \if@mid \dimen@\ht\z@ \advance\dimen@\dp\z@ \advance\dimen@12\p@
    \advance\dimen@\pagetotal \advance\dimen@-\pageshrink
    \ifdim\dimen@>\pagegoal\@midfalse\p@gefalse\fi\fi
  \if@mid \bigskip\box\z@\bigbreak
  \else\insert\topins{\penalty100 % floating insertion
    \splittopskip\z@skip
    \splitmaxdepth\maxdimen \floatingpenalty\z@
    \ifp@ge \dimen@\dp\z@
    \vbox to\vsize{\unvbox\z@\kern-\dimen@}% depth is zero
    \else \box\z@\nobreak\bigskip\fi}\fi\endgroup}

\output{\plainoutput}
\def\plainoutput{\shipout\vbox{\makeheadline\pagebody\makefootline}%
  \advancepageno
  \ifnum\outputpenalty>-\@MM \else\dosupereject\fi}
\def\pagebody{\vbox to\vsize{\boxmaxdepth\maxdepth \pagecontents}}

\newif\if@twoside \@twosidefalse

\def\pagestyle#1{\csname ps@#1\endcsname}
\newskip\headlineskip \headlineskip=-30.0mm
\newskip\footlineskip \footlineskip=5mm

\def\makeheadline{\vskip\headlineskip
                \line{\if@twoside\ifodd\@oddhead\else\@evenhead\fi\else\@oddhead\fi}\vskip5mm}
\def\makefootline{\baselineskip\footlineskip\lineskiplimit0pt
                \line{\if@twoside\ifodd\@oddfoot\else\@evenfoot\fi\else\@oddfoot\fi}}
%define empty layout
\def\ps@plain{
  \def\@oddhead{}
  \def\@evenhead{}
  \def\@oddfoot {}
  \def\@evenfoot{}}
\pagestyle{plain}

\def\dosupereject{\ifnum\insertpenalties>\z@ % something is being held over
  \line{}\kern-\topskip\nobreak\vfill\supereject\fi}

\def\pagecontents{\ifvoid\topins\else\unvbox\topins\fi
  \dimen@=\dp\@cclv \unvbox\@cclv % open up \box255
  \ifvoid\footins\else % footnote info is present
    \vskip\skip\footins
    \footnoterule
    \unvbox\footins\fi
  \ifr@ggedbottom \kern-\dimen@ \vfil \fi}
\def\footnoterule{\kern-3\p@
  \hrule width 2truein \kern 2.6\p@} % the \hrule is .4pt high


\count0=0


\message{hyphenation,}

\lefthyphenmin=2 \righthyphenmin=3 % disallow x- or -xx breaks
\input hyphen

\def\magnification{\afterassignment\m@g\count@}
\def\m@g{\mag\count@
  \hsize6.5truein\vsize8.9truein\dimen\footins8truein}

\def\loggingall{\tracingcommands\tw@\tracingstats\tw@
  \tracingpages\@ne\tracingoutput\@ne\tracinglostchars\@ne
  \tracingmacros\tw@\tracingparagraphs\@ne\tracingrestores\@ne
  \showboxbreadth\maxdimen\showboxdepth\maxdimen}
\def\tracingall{\tracingonline\@ne\loggingall}

\def\showhyphens#1{\setbox0\vbox{\parfillskip\z@skip\hsize\maxdimen\tenrm
  \pretolerance\m@ne\tolerance\m@ne\hbadness0\showboxdepth0\ #1}}

\normalbaselines\rm % select roman font
\nonfrenchspacing % punctuation affects the spacing
\catcode`@=12 % at signs are no longer letters

\def\fmtname{plain}\def\fmtversion{3.141592653} % identifies the current format



\message{grades,} %added

\catcode`@=11
\newcount\c@edu@Grades \c@edu@Grades=0

% Setup Grade
\def\setGrade#1#2{
\global\advance\c@edu@Grades by 1
  \expandafter\xdef\csname edu@Grade@Id\the\c@edu@Grades\endcsname{#1}
  \expandafter\xdef\csname edu@Grade@PP\the\c@edu@Grades\endcsname{#2}}

\def\pgrade{
\count0=0
\loop
\ifnum\count0 < \c@edu@Grades \advance\count0 by 1
  \csname edu@Grade@Id\the\count0\endcsname\space
  \csname edu@Grade@PP\the\count0\endcsname\newline
\repeat
}

% Calclulate Grades
\def\getGradeList{
  \ifnum\c@edu@Grades>0 
  %\eduTex@PointsName: \the\c@eduTex@points
  \begingroup
  \vskip9pt \count0=0
  \setbox0=\hbox{\vbox{\hbox{\@gradeName}\hbox{p}\hbox{\@pointsName}\hbox{\vbox to 18pt{}}}
    \loop \ifnum\count0 < \c@edu@Grades \advance\count0 by 1
      \expandafter\dimen0=\the\c@points pt
      \expandafter\dimen0=\csname edu@Grade@PP\the\count0\endcsname\dimen0
      \vrule\kern9pt\vbox{\hbox to 30pt{\csname edu@Grade@Id\the\count0\endcsname}\hbox{\csname edu@Grade@PP\the\count0\endcsname}\hbox{\@strip@pt\dimen0}\hbox {\vbox to 18pt{}}} 
      \repeat
      \vrule     
     }
  \dimen0=0.6pt
  {\setbox2=\hbox{}\wd2=\@strip@pt\dimen0\wd0 \ht2=\@strip@pt\dimen0\ht0 \dp2=\@strip@pt\dimen0\dp0\box2}
  \pdfsave\pdfsetmatrix{0.6 0 0 0.6}\rlap{\smash{\copy0}}\pdfrestore
  \endgroup
  \fi    
}


\message{sheets,} %added
\catcode`@=11

\newcount\c@points
\newcount\c@spage
\newcount\c@exercisepoints
\newcount\c@exercise
\newcount\c@subexercise
\newcount\c@sheet
\newcount\c@sheetloop
\newcount\c@lesson 
\newbox\b@exercise
\newbox\b@cexercise
\newif\if@solve\@solvefalse
\newif\if@solveloop\@solveloopfalse
\newif\if@points\@pointsfalse

\def\strutA#1#2{\vrule height#1 depth#2 width0pt}
\def\drawbox#1{\kern2pt\vbox{\hrule height 1pt\relax
  \hbox{\hbox{\vrule width 1pt\relax\vbox{\hbox to #1{\vbox to #1{}}}\vrule width 1pt\relax}}
  \hrule height 1pt\relax}\kern2pt}

\def\pageid{\@initial\space- \folio}

\def\ps@Worksheet{
  \def\@oddhead{\rm\baselineskip=12pt\hskip-25mm\useColor{Header}{
    \global\parindent=0pt
    \vbox to \headlineheight{ 
    \hbox to \pdfpagewidth{\kern-2mm
    \vbox {\hbox to 23mm{\hfil\useColor{Highlight1}{\scaleBox[5]{\@initial}}}}
    \vrule height 17mm depth 3mm width 0pt
    \vbox{\hbox to \hsize {\bflarge\kern-1mm\useColor{Highlight1}{\@topic}\hss}
    \hbox to \hsize {\kern0mm\@subtopic\hss}}
    \vbox{\noindent\@icon}\hss
    }\hrule}\hss}}
  \def\@evenhead{\vbox to 0pt{\hbox to \hsize{\vbox to 10pt{\noindent\topic: \hfil}}\hrule}}
  \def\@oddfoot{\rmsmall\baselineskip=10pt\hskip-25mm\useColor{Footer}{
    \vbox to \footlineheight{\kern5mm\hrule\kern5pt
    \hbox to \pdfpagewidth{\hfill\@url\kern1cm}
    \hbox to \pdfpagewidth{\hfill\pageid\kern1cm}}\hss}}
  \def\@evenfoot{\url\hfill}
}


\def\makeSheetTitle#1#2#3#4{
  \parindent=0pt
  \vbox{
  \kern5mm
  \hbox to \hsize{\kern-3mm\useColor{TitleColor}{
  \vbox{\hbox to \hsize{\kern-1mm\bfscbig\useColor{Highlight1}
  {#1}\hfill\kern3mm\rmsmall#3}%\@subject}
        \hbox to \hsize{\rmlarge\useColor{Highlight2}
  {#2}\hfill\rmsmall #4}}%\@date}}
        \hss}}  
  \kern1mm
  \hrule width \hsize 
  \kern5mm
  }
}

\long\def\exercise#1#2{
  \parindent=0pt \parskip=0pt \c@exercisepoints=0 \c@subexercise=0
  \global\advance\c@exercise by 1
  \global\c@exercisepoints=0
  \if@solve\vbox{\setbox\b@exercise\hbox to \hsize{\vtop {#2\hss}\hss}\kern2mm
    \hbox to \hsize{\bf\@exerciseName\space\the\c@exercise:\space#1\hfill\if@points(\the\c@exercisepoints\space \@pointsName)\fi}\box\strut\b@exercise\kern0mm}   
  \else\vtop{\setbox\b@exercise\hbox to \hsize{\vtop {#2\hss}\hss}
    \kern2mm\hbox  to \hsize{\bf\@exerciseName\space\the\c@exercise:\space#1\hfill\if@points(\if@solveloop \kern1cm /\space\fi\the\c@exercisepoints\space \@pointsName)\fi\strut}
      \box\b@exercise\kern0mm}   
  \fi
  \par
  \vskip3mm
  \global\advance\c@points by \the\c@exercisepoints}


% Subexercise
\def\subexercise{\@ifstar \subexercisewstar \subexercisenstar}
\def\subexercisenstar{\@optarg\subexercisenwarg\subexercisennarg}
\long\def\subexercisennarg#1#2{\subexercisenwarg[1.0\hsize]{#1}{#2}}
\long\def\subexercisewstar*#1#2{\parindent=0pt\leavevmode\hbox to \hsize{\vtop{\if@solveloop#2\par\else\setbox\b@cexercise\vbox{#2}#1\fi\strut}\hfil\par}}
\long\def\subexercisenwarg[#1]#2#3{
  \advance\c@subexercise by 1\parindent=0pt \parskip=0pt
  \leavevmode\hbox to #1{ \hsize=#1
    \advance\hsize by -10mm
    \vtop{\hbox to 5mm{\bf\alphanumeral\c@subexercise)\hss}}
    \vtop{\hbox to \hsize{\vtop{\if@solveloop#3\else\setbox\b@cexercise\vbox{#3}#2\fi\strut\par}\hfil}}\hss}}

\def\ap{\space{\drawbox{2mm}}\global\advance\c@exercisepoints by 1\relax}

\def\newSeries#1#2{
\def\@topic{#1}
\def\@initial{#2}
\c@sheet=0
}


\long\def\checkTable#1{
  \long\def\top##1##2{##1&##2&&&\cr}
  \def\selbox##1{\vtop{\hbox to 5mm{\hfil##1\hfil}\hbox{\drawbox{3mm}}}}
  \def\selboxes{  \hbox{\selbox{\bf++}\selbox{\bf+}\selbox{\bf{o}}\selbox{\bf-}}}
  \everycr{\noalign{\hrule}}
  \halign{
    \vrule\strutA{10pt}{4pt}
    \hfil\kern5pt\hbox to 0.25\hsize{\hsize=0.2\hsize\vtop{##\strut\par\selboxes\strutA{1pt}{1pt}}}\kern5pt\hfil\vrule&
    \hfil\kern5pt\hbox to 0.23\hsize{\hsize=0.2\hsize\vtop{##}}\kern5pt\hfil\vrule&
    \hfil\kern5pt\hbox to 0.1\hsize{\vbox{##}}\kern5pt\hfil\vrule&
    \hfil\kern5pt\hbox to 0.2\hsize{\vbox{##}}\kern5pt\hfil\vrule&
    \hfil\kern5pt\hbox to 0.1\hsize{\vbox{##}}\kern5pt\hfil\vrule\cr
    \multispan3\vrule\strutA{12pt}{5pt}\hfil \bf Vor der Arbeit\hfil\vrule& \multispan2\hfil \bf Nach der Arbeit\hfil\vrule\cr
    \omit\vrule\strutA{10pt}{4pt}\hfil\vrule& Aufgaben & Geübt? & Empfehlung & Geübt?\cr
    #1}}

\long\def\newCList#1#2{
\newpage
\def\@subtopic{#1}
\pagestyle{Worksheet}
  \makeSheetTitle{#1}{Rückmeldebogen}{ }{\rmsmall\@nameName:\kern50mm}
#2
  \vskip2cm
  \hbox to \hsize{\hfil\hsize=55mm
    \vtop{\hrule \sevenrm Unterschrift Erziehungsberechtigte/-r\strut}\kern5mm
    \vtop{\hrule \sevenrm Unterschrift Erziehungsberechtigte/-r\strut}}
\newpage
}

\long\def\newPlanning#1#2#3{
\c@lesson=1
\long\def\lesson##1##2{\vskip2mm
\hbox{\hbox to 0.1\hsize{\bf\the\c@lesson\hfil}
\hbox to 0.9\hsize{\hsize=0.9\hsize\vtop{{\bf##1}\newline##2}}}
\advance\c@lesson by 1}
\long\def\objectives##1{
\vskip2mm
\hrule
\vskip2mm
##1
\vskip2mm
\hrule
\vskip2mm
}
\newpage
\def\@subtopic{Reihenplanung}
\c@points=0
\c@exercise=0
\pagestyle{Worksheet}
\makeSheetTitle{Reihenplanung}{#1}{ }{#2}
#3
\newpage
}

\long\def\newWorksheet#1#2{
\newpage
\def\@subtopic{#1}
\c@points=0
\c@exercise=0
\pagestyle{Worksheet}
\advance\c@sheet by 1
\makeSheetTitle{\@worksheetName\space\the\c@sheet}{#1}{}{\@nameName:\kern50mm}
#2
\newpage
}

\long\def\newTest#1#2{
\newpage
\def\@subtopic{\@testName}
\c@points=0
\c@exercise=0
\pagestyle{Worksheet}
\begingroup
\@pointstrue

\global\c@points=0
\c@exercise=0
\makeSheetTitle{\@testName}{#1}{}{\@nameName:\kern50mm}
#2
\vskip1cm
{\hfill\bf \@wishName \kern3cm(\the\c@points\space\@pointsName)}
\newpage

\@solvelooptrue
\global\c@points=0
\c@exercise=0
\makeSheetTitle{\@testName}{#1}{}{\@solutionName}
#2
\vskip1cm
\hfill{\bf\@totalName:\kern2cm / \the\c@points\space\@pointsName}
\newpage
\@solveloopfalse
\endgroup
}

\long\def\newExam#1#2{
\newpage
\def\@subtopic{\@examName}
\pagestyle{Worksheet}
\begingroup
\@pointstrue

\global\c@points=0
\c@exercise=0
\makeSheetTitle{\@examName}{#1}{}{\@nameName:\kern50mm}
#2
\vskip1cm
{\hfill\bf \@wishName \kern3cm(\the\c@points\space\@pointsName)}
\newpage

\@solvelooptrue
\global\c@points=0
\c@exercise=0
\makeSheetTitle{\@examName}{#1}{}{\@solutionName}
#2
\vskip1cm
\hfill{\bf\@totalName:\kern2cm / \the\c@points\space\@pointsName}
\getGradeList
\newpage
\@solveloopfalse

\endgroup
}


\catcode`@=12
\dump
\endinput
